%% Preambulo
\documentclass[12pt, a4paper]{article}
\usepackage[utf8]{inputenc}
\usepackage{amsmath}

%% Ajustes

%% Personalizaciones
\makeatletter

\makeatother

%% Documento
\begin{document}
\title{Paquete \texttt{amsmath}}
\author{Martin Vuelta}
\maketitle

\section{Entorno de ecuaciones}

\subsection{Entorno \texttt{equation}}
\begin{verbatim}
\begin{equation}
    m \frac{d^{2} x}{d t^{2}} = - k x
\end{equation}
\end{verbatim}

\begin{equation}
    m \frac{d^{2} x}{d t^{2}} = - k x
\end{equation}

\subsection{Entorno \texttt{equation*}}
\begin{verbatim}
\begin{equation*}
    m \frac{d^{2} x}{d t^{2}} = - k x
\end{equation*}
\end{verbatim}

\begin{equation*}
    m \frac{d^{2} x}{d t^{2}} = - k x
\end{equation*}

\subsection{Entorno \texttt{align}}
\begin{verbatim}
\begin{align}
    (a + b)^2 &= (a + b)(a + b) \nonumber \\
              &= a^2 + ab + ba + b^2 \\
              &= a^2 + 2ab + b^2 \nonumber
\end{align}
\end{verbatim}

\begin{align}
    (a + b)^2 &= (a + b)(a + b) \nonumber \\
              &= a^2 + ab + ba + b^2 \\
              &= a^2 + 2ab + b^2 \nonumber
\end{align}

\subsection{Entorno \texttt{gather}}
\begin{verbatim}
\begin{gather}
    a^2 + b^2 = c^2 \\
\sin^2(\theta) + \cos^2(\theta) = 1
\end{gather}
\end{verbatim}

\begin{gather}
    a^2 + b^2 = c^2 \\
\sin^2(\theta) + \cos^2(\theta) = 1
\end{gather}


\subsection{Entorno \texttt{multiline}}
\begin{verbatim}
\begin{multline}
\framebox[0.65\columnwidth]{A} \\
\shoveleft{\framebox[0.50\columnwidth]{B}} \\
\shoveright{\framebox[0.55\columnwidth]{C}} \\
\framebox[0.65\columnwidth]{D}
\end{multline}
\end{verbatim}


\begin{multline}
\framebox[0.65\columnwidth]{A} \\
\shoveleft{\framebox[0.50\columnwidth]{B}} \\ 
\shoveright{\framebox[0.55\columnwidth]{C}} \\
\framebox[0.65\columnwidth]{D}
\end{multline} 

\subsection{Entorno \texttt{split}}
\begin{verbatim}
\begin{equation}
    \begin{split}
        H_{c} & = \frac{1}{2n} \sum_{l=0}^{n} (-1)^{l}
        (n - l)^{p} \sum_{l_{1} + \dots + l_{p}=l}
        \prod_{i=1}^{p} \binom{n_i}{l_i} \\
        & \quad \cdot [(n-l) - 
        (n_{i} - l_{i})]^{n_{i} - l_{i}}
        \cdot \left[(n - l)^{2} - 
        \sum_{j=1}^{p} (n_{i} - l_{i})^2 \right]
    \end{split}
\end{equation}
\end{verbatim}

\begin{equation}
    \begin{split}
        H_{c} & = \frac{1}{2n} \sum_{l=0}^{n} (-1)^{l} (n - l)^{p}
        \sum_{l_{1} + \dots + l_{p}=l} \prod_{i=1}^{p} \binom{n_i}{l_i} \\
              & \quad \cdot [(n-l) - (n_{i} - l_{i})]^{n_{i} - l_{i}}
        \cdot \left[(n - l)^{2} - \sum_{j=1}^{p} (n_{i} - l_{i})^2 \right]
    \end{split}
\end{equation}

\subsection{Entorno \texttt{array}}
\begin{verbatim}
\begin{equation}
    \left(
        \begin{array}{c c c}
            1 & 2 & 3 \\
            4 & 5 & 6 \\
            7 & 8 & 9
        \end{array}
    \right)
    \times
    \left(
        \begin{array}{c c c}
            1 & 2 & 3 \\
            4 & 5 & 6 \\
            7 & 8 & 9
        \end{array}
    \right) 
\end{equation}
\end{verbatim}

\begin{equation}
    \left(
        \begin{array}{c c c}
            1 & 2 & 3 \\
            4 & 5 & 6 \\
            7 & 8 & 9
        \end{array}
    \right)
    \times
    \left(
        \begin{array}{c c c}
            1 & 2 & 3 \\
            4 & 5 & 6 \\
            7 & 8 & 9
        \end{array}
    \right) 
\end{equation}

\subsection{Entorno \texttt{cases}}

\begin{equation}
    \delta_{ij} = \begin{cases}
    0 & i \neq j \\
    1 & i = j
    \end{cases}
\end{equation}

\begin{equation}
    \delta_{ij} = \left\{
    \begin{array}{c c}
        0 & i \neq j \\
        1 & i = j
    \end{array}
    \right.
\end{equation}

\end{document}
