%% Preambulo
\documentclass[a4paper, 12pt, final]{examen}
\usepackage[utf8]{inputenc}
\usepackage[spanish]{babel}
\usepackage{graphicx}
\usepackage{hyperref}
\usepackage{charter}
\usepackage{lipsum}

%% Configuraciones
% Configuraciones del paquete HyperRef
\hypersetup{
    pdftitle={Sesión 3},
    pdfauthor={Martín Vuelta},
    pdfpagelayout=OneColumn,
    pdfnewwindow=true,
    pdfdisplaydoctitle=true,
    unicode=true,
    bookmarksnumbered=true,
    bookmarksopen=false,
    breaklinks=true,
    colorlinks=true
}

% Configuraciones del paquete graphicx
\graphicspath{{resources/img/}}

%% Personalización de Usuario
\makeatletter

% Personalización de fuentes
% \rmdefault : roman (serif)
% \sfdefault : sans serif
\renewcommand{\familydefault}{\rmdefault}

\makeatother

%% Cuerpo del documento
\begin{document}
	\logo{upla.jpg}
	\university{Universidad Nacional del Altiplano}
    \course{Curso de \LaTeX}
    \title{Práctica Calificada}
    \author{Martin Vuelta}
    \date{\today}
    \maketitle
    
    \section{Seccion 1}\label{sec:1}
        \lipsum[1-4]
        
        Esta oración tendrá un pié de página.\footnote{\lipsum[2]} Y esta oración tendrá una referencia hacia el título de la sección  \ref{sec:1}. Este texto aparecerá en formato de \texttt{ancho fijo}.

\end{document}
