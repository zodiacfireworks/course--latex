\documentclass{beamer}
\usepackage[utf8]{inputenc}
\usepackage[spanish]{babel}
\usepackage[T1]{fontenc}
\usepackage{pdflscape}
\usepackage{epstopdf}
\usepackage{fancyhdr}
\usepackage{fancyvrb}
\usepackage{graphicx}
\usepackage{hyperref}
\usepackage{luximono}
\usepackage{multicol}
\usepackage{rotating}
\usepackage{tabularx}
\usepackage{textcomp}
\usepackage{amsmath}
\usepackage{amssymb}
\usepackage{amstext}
\usepackage{charter}
\usepackage{amsbsy}
\usepackage{amsthm}
\usepackage{lipsum}
\usepackage{minted}
\usepackage{natbib}
\usepackage{array}
\usepackage{color}
\usepackage{float}


%% Settings
%% ----------------------------------------------------------------------
%% Hyperref package settings
\hypersetup{
    pdftitle={Prueba de estrés y rendimiento para el Servidor de Microsoft Dynamics NAV\textregistered. Instalaci\'{o}n de Norma},
    pdfauthor={Mart\'{i}n Josemar\'{i}a Vuelta Rojas},
    pdfpagelayout=OneColumn,
    pdfnewwindow=true,
    pdfdisplaydoctitle=true,
    pdfstartview=XYZ,
    plainpages=false,
    unicode=true,
    bookmarksnumbered=true,
    bookmarksopen=true,
    bookmarksopenlevel=3,
    breaklinks=true,
    colorlinks=true,
    pdfborder={0 0 0}
}

%% Graphicx Settings
\graphicspath{{resources/img/}}

%% Beamer Package Settins
\usefonttheme{structuresmallcapsserif}
\usetheme{Madrid}
\usecolortheme{crane}

% gets rid of bottom navigation bars
% \setbeamertemplate{footline}[frame number]{}
% gets rid of bottom navigation symbols
\setbeamertemplate{navigation symbols}{}
% gets rid of footer
% will override 'frame number' instruction above
% comment out to revert to previous/default definitions
% \setbeamertemplate{footline}{}


%% Command definitions
%% ----------------------------------------------------------------------
\makeatletter

%% Color definitions
\definecolor{red-softbutterfly}{rgb}{0.7843137254901961, 0.21568627450980393, 0.21568627450980393}
\definecolor{blue-softbutterfly}{rgb}{0.08627450980392157, 0.17647058823529413, 0.3137254901960784}
\definecolor{red-A700}{rgb}{1.0000, 0.0000, 0.0000}
\definecolor{grey-050}{rgb}{0.9804, 0.9804, 0.9804}
\definecolor{grey-800}{rgb}{0.2588, 0.2588, 0.2588}
\definecolor{grey-900}{rgb}{0.1216, 0.1216, 0.1216}

\newcommand{\globalcolor}[1]{\color{#1}\global\let\default@color\current@color}

%% Table columns types
\newcolumntype{L}[1]{>{\raggedright\let\newline\\\arraybackslash\hspace{0pt}}m{#1}}
\newcolumntype{C}[1]{>{\centering\let\newline\\\arraybackslash\hspace{0pt}}m{#1}}
\newcolumntype{R}[1]{>{\raggedleft\let\newline\\\arraybackslash\hspace{0pt}}m{#1}}

%% Serif type font as default for all document
\renewcommand{\familydefault}{\rmdefault}

%% Registered mark text
\newcommand{\registeredmark}{\textsuperscript{\small{\textregistered}}}
\makeatother


%% Document
%% ----------------------------------------------------------------------
\begin{document}

\title[\LaTeX]{Introducción a \LaTeX}
\subtitle{Conceptos básicos}
\author[@zodiacfireworks]{Martín~Vuelta\inst{1}\inst{2}}

\institute[]{
	\inst{1}%
	Facultad de Ciencias Físicas\\
	Universidad Nacional Mayor de San Marcos
	\and
	\inst{2}%
	Dirección de Desarrollo de Software\\
	SoftButterfly
}

% \logo{%
%    \includegraphics[height=0.5cm]{SoftButterfly-LaTeX-Logo.pdf}%
% }

\date{\today}
\frame{\titlepage}

\section{Contenidos}
    \begin{frame}
    \frametitle{Table of Contents}
    \tableofcontents
    \end{frame}

\section{{\TeX} \& {\LaTeX}}
    \begin{frame}
        \frametitle{{\TeX} \& {\LaTeX}}

        \begin{block}{{\TeX}}
            {\TeX} es un lenguaje de computadora diseñado para ser usado en la tipografía; En particular, para la composición de matemáticas y otras materias técnicas.
        \end{block}

        \href{https://www.tug.org/whatis.html}{https://www.tug.org/whatis.html}
    \end{frame}

    \begin{frame}
        \frametitle{{\TeX} \& {\LaTeX}}

        \begin{block}{{\LaTeX}}
            {\LaTeX} es un sistema de preparación de documentos para tipografía de alta calidad. Se utiliza con mayor frecuencia para documentos técnicos o científicos de tamaño medio a grande, pero puede utilizarse para casi cualquier forma de publicación.
        \end{block}

        \href{http://www.latex-project.org/about/}{http://www.latex-project.org/about/}
    \end{frame}

\section{Estructura de un documento en {\LaTeX}}
    \begin{frame}
        \frametitle{Estructura de un documento en {\LaTeX}}
        \inputminted{text}{resources/snippets/document_structure.txt}


    \end{frame}
\end{document}
