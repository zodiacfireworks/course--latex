% Preambulo -----------
% Tipo de documento
\documentclass[a4paper]{article}

% Paquetes
% \usepackage[OPCIONES]{NOMBBRE DEL PAQUETE}

% Inputenc: Input encoding
% Codificación de texto, emplear preferibemente
% utf-8
\usepackage[utf8]{inputenc}

% Babel: Paquete para traducciones automáticas
\usepackage[spanish]{babel}

% Graphicx: Paquete para la iserción de imágenes
\usepackage{graphicx}

% Wrapfig: Paquete para envolver la figura con el texto
\usepackage{wrapfig}

% Lipsum: Paquete para texto de relleno
\usepackage{lipsum}



% Cuerpo del documento -----
\begin{document}
    \title{\LaTeX: Sesión 1}
    \author{Martin Vuelta}
    % \date{17 de Junio del 2016}
    \date{\today}
    \maketitle

    \section{Uso de tildes}
        Uso de tildes sin \texttt{inputenc}

        \begin{verbatim}
        Alg\'{u}n texto: \'{e} \'{i}
        \end{verbatim}

        Uso de tildes con \texttt{inputenc}

        \begin{verbatim}
        Algún texto: é í
        \end{verbatim}

    \section{Uso de entornos}\label{sec:2}

        Un entorno tiene la siguiente sintaxis

        \begin{verbatim}
            \begin[OPCIONES]{nombre del entorno}
            ... contenido ...
            \end{nombre del entorno}
        \end{verbatim}

    \section{Uso de imágenes}
        Las imágenes se insertan haciendo uso del paquete \texttt{graphicx}

        Primer uso: etiqueta \texttt{includegraphics}

        \includegraphics[scale=0.5]{imagen.jpg}

        \begin{center}
            \includegraphics[height=3cm]{imagen.jpg}
        \end{center}

        \includegraphics[width=\textwidth]{imagen.jpg}

        \begin{figure}
            \centering
            \includegraphics[
                width=0.5\textwidth,
                angle=45
            ]{imagen.jpg}
            \caption{Chica de anime con gafas rojas}
            \label{fig:1}
        \end{figure}

    \section{Uso de Lipsum e imágenes alineadas}
        \lipsum[1-2]

        \begin{wrapfigure}{l}{0.25\textwidth}
            \centering
            \includegraphics[
                width=0.25\textwidth
            ]{imagen.jpg}
        \end{wrapfigure}

        \lipsum[3-4]

    \section{Referencias internas}
        Referencia a imagen: \emph{fig.} \ref{fig:1}

        Referencia a sección: \textit{sec.} \ref{sec:2}

        \emph{Donec ornare nulla sed dictum luctus. \emph{Aenean} et dui vel orci posuere volutpat sit amet sed est}.
\end{document}
